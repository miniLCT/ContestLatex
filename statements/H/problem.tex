\begin{problem}{LONG LONG LONG LONG Winter Vacation}{standard input}{standard output}{1 second}{512 megabytes}
$Zhuangba$ 是一个非常自律的高三学生,在没有开学的时候也能做到很好的时间管理。由于在浙江首轮高考中取得不错的成绩,她只需要复习语文,数学,英语和强基物理。
    
现在假设她有长达 $N$ 天的寒假,每一天她都会复习语文,数学,强基物理中的一门,获得该门学课的成就感(在第 $i$ 天,复习语文获得 $a_i$的成就感,复习数学 $b_i$的成就感,复习英语获得 $c_i$的成就感,复习强基物理获得 $d_i$ 的成就感)。从第二天开始,她复习的内容和昨天的内容是不一样的,即如果第一天复习了数学,那么第二天只能复习语文或英语或强基物理。
    
擅长管理时间的 $Zhuangba$ 很快就计算出如何安排能够获得最大成就感。
\InputFile
第一行输入寒假的天数 $N(1\le N\le 10^6)$。

接下来 $N$ 行每一行输入仅由空格隔开的四个值$a_i,b_i,c_i,d_i$表示第 $i$ 天复习语文,数学,英语,强基物理所能获得的成就感$(1\le a_i,b_i,c_i,d_i\le 10^5)$。

\OutputFile

输出 $Zhuangba$ 安排获得的最大成就感的值。

\Example

\begin{example}
\exmpfile{example.1.in}{example.1.ans}%
\end{example}

\end{problem}
