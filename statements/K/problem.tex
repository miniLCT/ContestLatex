\begin{problem}{Date}{standard input}{standard output}{1 second}{512 megabytes}

rkmdsxmds 一直想和 Zhuangba 约恰饭,但是 Zhuangba 非常的忙,只有在 "Perfect Day" 时候 Zhuangba 才会同意和 rkmdsxmds 约恰饭。

Zhuangba 定义的 "Perfect Day" 是当天有个时间构成回文序列。比如 2020 年 02 月 22 日 20:02:02,是一个回文序列(20200222200202)。这一天 Zhuangba 同意和 rkmdsxmds 恰饭。

于是 $rkmdsxmds$ 非常非常期待着每一个 "Perfect Day",他很想知道 2020 年 02 月 22 日 20:02:02 后的第 $k$ 个 "Perfect Day"是什么。

\InputFile

第一行输入测试样例 $t(1\le t \le 1000)$。 

接下来 $t$ 行,每行一个非负整数 $k$ ,要求第 $k$ 个"Perfect Day"。 

\OutputFile

输出 $t$ 行,每一行一个字符串表示答案。

\Example

\begin{example}
\exmpfile{example.1.in}{example.1.ans}%
\end{example}

\Note
特别的,认为2020 年 02 月 22 日 20:02:02是之后的第 0 个"Perfect Day"。

保证对于输入,输出答案不会大于 99991231235959 ,即答案都早于 10000 年 1 月 1 日。输出的答案需要符合时间的规范,用0补全。

\end{problem}
