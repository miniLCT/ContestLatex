\begin{problem}{Zhuangba and rkmdsxmds}{standard input}{standard output}{2 second}{256 megabytes}

% \includegraphics[width=0.35\textwidth]{ad.png}%

Zhuangba 和 rkmdsxmds 常常会在一起玩游戏。

有一天他们在一起玩“翻格格”的游戏,游戏规则是这样的:

Zhuangba 和 rkmdsxmds 面前有一个二维平面意义下的 $n\times m$ 的格子图,格子被认为有正面反面,刚开始所有的格子都是正面朝上。翻动一个格子后,会把此格子作为中心(它自身也要翻面)带动着这个格子的九宫格(如果相邻格子存在且合法)翻面(即原本正面朝上变为反面朝上,原本反面朝上变为正面朝上)。规定只有所有格子作为中心翻动一次后游戏结束。

当游戏结束后如果反面朝上的数量是偶数个,认为 Zhuangba 获胜,否则 rkmdsxmds 获胜。

Zhuangba 和 rkmdsxmds 都会用最优策略翻格子,请你判断一下是谁获胜了。
\InputFile

% \includegraphics{fuck.jpg}

第一行一个整数 $t 
(1\le t \le 1000)$。

接下来 $t$ 行,每行两个整数$n , m (1\le n , m \le 10^5)$
\OutputFile
输出 $t$ 行,如果 $Zhuangba$ 获胜,输出"Zhuangba\ txdy",否则输出"Zhuangba MeiMeiMei"。

\Example

\begin{example}
\exmpfile{example.1.in}{example.1.ans}%
\end{example}

% \Notes

% \includegraphics[width=0.80\textwidth]{ad.png}%
% \includegraphics{fuck.eps}


\end{problem}
