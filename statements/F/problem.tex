\begin{problem}{Zhuangba and Dijkstra}{standard input}{standard output}{2 second}{512 megabytes}

rkmdsxmds 在学了 Dijkstra 算法后很高兴,马上就和 Zhuangba 分享。

Zhuangba 想考验一下 rkmdsxmds 是否真的理解了,于是丢给他一个题:

有 $n$ 个星球,编号是 $1$ 到 $n$,rkmdsxmds 和 Zhuangba 各种不可描述原因分隔到两个不同星球。rkmdsxmds 被留在了 $s$ 星球,而 Zhuangba 不知道去了哪里。

现在一共有三种方案:

\begin{itemize}
\item v 星球到 u 星球,花费w体力。
\item v 星球到 $\left[l, r\right]$ 区间范围内的一个星球,花费 w 体力。
\item $\left[l, r\right]$ 区间范围内星球到达 v 星球,花费 w 体力。
\end{itemize}

因为rkmdsxmds 不知道 Zhuangba 在哪里,所以并且他想要赶快找到通往所有星球的道路各一条并立刻出发。因此对于每一个星球(包括 $s$ 星球本身)他想要知道从$s$星球到这个星球所用的最小体力值。

\InputFile

输入数据的第一行包括三个整数 $n$,$q(1\le n , q \le 10^5)$ 和 $s(1\le s \le n)$ 分别表示星球的数目,可行操作数以及$rkmdsxmds$所在星球的编号。

接下来的 $q$ 行表示 $q$ 种方案。

\begin{itemize}
\item 输入 $1\ v\ u\ w$  表示第一种方案,其中 $v,u(1\le v, u\le 10^5)$ 意思同上描述,$w(1\le w \le 10^9)$ 表示此方案所需花费最小的体力。
\item 输入 $2\ v\ l\ r\ w$ 表示第二种方案,其中 $v,l,r(1\le v , l , r \le 10^5)$ 意思同上描述,$w(1\le w \le 10^9)$ 表示此方案所需花费最小的体力。
\item 输入 $3\ v\ l\ r\ w$ 表示第三种方案,其中 $v,l,r(1\le v , l , r \le 10^5)$ 意思同上描述,$w(1\le w \le 10^9)$ 表示此方案所需花费最小的体力。
\end{itemize}

\OutputFile

输出一行用空格隔开的 $n$ 个整数分别表示从 $s$ 星球到第 $i$ 个星球所需的最少消耗体力值。

\Example

\begin{example}
\exmpfile{example.1.in}{example.1.ans}%
\end{example}

\Notes

rkmdsxmds 可以先购买第 4 个方案再购买第 2 个方案从而到达标号为 2 的星球.

\end{problem}
