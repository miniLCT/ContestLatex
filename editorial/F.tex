\renewcommand{\problemname}{F. Plant tree?}

\begin{frame}\frametitle{\problemname}

    \begin{block}{题意}
        种树获得快乐值,快乐值是严格递增的。且满足一些限制。
		\begin{itemize}
			\item[$\cdot$] 保证相邻两项差值 $\le Y$
			\item[$\cdot$] 保证最大的快乐值 $\le Z$
		\end{itemize}
		求可能的方案数。
    \end{block}

    \pause

注意到及其鬼畜的数据范围。

\end{frame}

\begin{frame}\frametitle{\problemname}
	
	\begin{block}{题解}
		{
			\small
		把方案数想成是数组滑动一样的东西,感性的理解一下。

		这样理解之后好像没有怎么用。。

		连续X天,差值最大为Y,最大值为Z

        不妨令第 $i$ 天与第 $i+1$ 的差值为 $a_i$	

		$$ans=\sum_{a_1=1}^Y\sum_{a_2=1}^Y\cdots\sum_{a_{X-1}=1}^Y(Z-a_1-a_2\cdots-a_{X-1})$$
		对于 $Z$ 部分求和就是
		$$Z\cdot Y^{X-1}$$
		接下来考虑后面的部分
		$$
		\begin{aligned}
		right&=\sum_{a_1=1}^Y\sum_{a_2=1}^Y\cdots\sum_{a_{X-1}=1}^Y(a_1+a_2\cdots+a_{X-1})\\
		\end{aligned}
		$$
		}
	\end{block}
\end{frame}


\begin{frame}\frametitle{\problemname}
	
	\begin{block}{F. Plant tree?}
		{
			\small
		考虑每个数的贡献:该数的每个取值对应贡献是 $Y^{X-2}$.
		于是有:
		{
			\tiny
		$$
		\begin{aligned}
		right1&=\sum_{a_1=1}^Y\sum_{a_2=1}^Y\cdots\sum_{a_{X-1}=1}^Y(a_1+a_2\cdots+a_{X-1})\\
		&=\sum_{i=1}^Y Y^{X-2}\\
		&=\frac{(Y+1)\cdot Y^{X-1}}{2}
		\end{aligned}
		$$
		}
		所以,总和就是
		{
			\tiny
		$$
		right=(X-1)\cdot right1=\frac{(X-1)\cdot(Y+1)\cdot Y^{X-1}}{2}
		$$
		}
		最后,合并一下答案
		{
			\tiny
		$$
		\begin{aligned}
		ans&=Z\cdot Y^{X-1}-\frac{(Y+1)\cdot Y^{X-1}}{2}\\
		&=\frac{Y^{X-1}\cdot(2\cdot Z-(X-1)(Y+1)) }{2}
		\end{aligned}
		$$	
		}
		}
	\end{block}

\end{frame}
\begin{frame}\frametitle{\problemname}
\begin{block}{花絮}
		这道题在今天9点左右被提出有问题,原因是我没有添加这样的条件 $(X-1)\times Y \le Z$。这样的话以上所有的式子都是不成立的。于是某个出题人建议把 $Z$ 的范围改成大于 1e12 ,于是有了现在这么鬼畜的数据范围。
	\end{block}
		
\end{frame}



