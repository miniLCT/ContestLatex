\renewcommand{\problemname}{G. An Easy Math Problem}

\begin{frame}\frametitle{\problemname}
	
	\begin{block}{题意}
		构造两个乘积相等的等差数列(数两两不同)。
	\end{block}
\end{frame}

\begin{frame}\frametitle{\problemname}

	\begin{block}{想法}
		样例是来迷惑人的,偶数时候可以如同样例那样构造,但在奇数时候却不可以。

		打表5,7,发现规律。
		
		\begin{bmatrix}
    	10&72&134&196&258\\
    	36&67&98&129&160\\
		\end{bmatrix}

		\begin{bmatrix}
    	14&268&522&776&1030&1284&1538\\
    	134&261&388&515&642&769&896\\
		\end{bmatrix}
	\end{block}

	\begin{block}{花絮}
		学长说网络赛嘛,出个打表题,打个四五小时的表,然后偶然间看到中等数学上一个等差数列的题目,然后想着魔改一下。开始不确定这样有无解,后来打着打着,发现这个形式很像有规律,然后做做差分发现可行解了。
	\end{block}
	
\end{frame}